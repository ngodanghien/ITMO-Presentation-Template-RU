\begin{frame}{Актуальность}
	\hypertarget{slide\insertframenumber}{}	
	Современные технические системы наиболее точно могут быть описаны нестационарными математическими моделями, в которых параметры могут изменяться во времени.
	
	\vspace{5mm}
	
	\begin{figure}[!h]
		\begin{subfigure}[t]{0.3\textwidth}
			\centering
			%\includegraphics[width=0.83\linewidth]{images/AM.jpg}
			\includegraphics[width=0.5\linewidth]{example-image} 
			\caption{Модель асинхронного двигателя}
		\end{subfigure}
		\begin{subfigure}[t]{0.3\textwidth}
			\centering
			\includegraphics[width=0.8\textwidth]{example-image} 
			\caption{Системы позиционирования судна}
		\end{subfigure}
		\begin{subfigure}[t]{0.3\textwidth}
			\centering
			\includegraphics[width=0.8\textwidth]{example-image} 
			\caption{Модель динамики манипулятора}
		\end{subfigure}
		\vspace{2mm}
		\caption*{Примеры систем с переменными параметрами}
	\end{figure}
\end{frame}

\section*{Цель, задачи исследования}
\setfootlinetext{Цель, задачи исследования}
\begin{frame}{Цель, задачи} \hypertarget{slide\insertframenumber}{}	
	\textbf{Цель исследования.}
	Целью диссертационной работы является синтез законов адаптивного управления для класса нестационарных систем в условиях параметрической неопределенности постоянной матрицы состояния и переменной матрицы входов.
	
	\vspace*{2mm}
	\textbf{Задачи.}
	%Для достижения данной цели в рамках диссертации были поставлены и решены следующие задачи:
	\begin{enumerate}
		\item Синтез закона адаптивного управления по выходу для линейного нестационарного объекта, модель которого содержит переменные параметры, описываемые управляемым генератором с известной матрицей состояния.
		\vspace*{2mm}
		\item Синтез адаптивного наблюдателя производных выходной регулируемой переменной для нестационарного объекта с параметрически неопределенной моделью переменных элементов матрицы входа.
		\vspace*{2mm}
		\item Синтез алгоритма адаптивного управления по выходу для класса нестационарных систем в условиях параметрической неопределенности с приложением для асинхронного двигателя с неизвестными сопротивлением, индуктивностью и моментом нагрузки.
	\end{enumerate}
\end{frame}